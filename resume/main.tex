\documentclass[11pt,a4papersans]{moderncv}
\usepackage{lmodern, textcomp}
\moderncvtheme[blue]{classic}
\usepackage{csquotes}
\usepackage{comment}
\usepackage[T1]{fontenc}
% \usepackage[
%   backend=biber,
%   style=numeric,
%   date=year,
%   sortcites,
%   sorting=none,
%   url=false,
%   maxbibnames=99,
%   ]{biblatex}
% \usepackage{geometry}
\usepackage[scale=0.9]{geometry}

\firstname{Yann}
\familyname{Labbé}
%\address{54 Rue Maurice Ripoche}{75014, Paris}
\homepage{ylabbe.github.io}
\social[github]{ylabbe}
\extrainfo{\emailsymbol\emaillink{labbe.yann1994@gmail.com}}
%\mobile{+33 6 58 81 77 30}
%\extrainfo{31 years old}
\nopagenumbers{}


\begin{document}
\makecvtitle
\vspace{-4em}
\section{Experience}
\cventry{July 2024 --- June 2025}{H Company}{Paris}{Research Scientist}{}{
In the founding team of a startup specialized on AI agents.
\begin{itemize}
    \item Training VLMs for web agents. 
    \item Building the research infrastructure. 
    %\item  Scaling the company from 10 to 60 employees through interview and mentorship activities.
\end{itemize}
}
\cventry{June 2023 --- June 2024}{Meta}{Zurich}{Research Scientist}{}{
In the team that develops and ships hand tracking algorithms for the Meta Quest headsets.
  \begin{itemize}
    \item Training, evaluation and simulation of hand pose forecasting models.
    %\item Intern peer, author of a CVPR publication on 6D object pose estimation.
    %\item Co-organized the R6D Workshop and BOP challenge at ICCV 2023.
  \end{itemize}
}

\cventry{Feb. 2022 --- May 2022}{Nvidia}{Remote}{Research intern}{}{
In a team focused on robotics.
%\begin{itemize}
%    \item First author of a CoRL publication with 10 authors.
%\end{itemize}
  % \begin{itemize}
  % \item 
  % % \item In 2024, the open-source method MegaPose is used in the Atlas robot at Boston Dynamics to perform object manipulation.
  % \end{itemize}
}{
}

\cventry{2019 --- 2022}{École Normale Supérieure}{Paris}{Teaching assistant }{}{``Object recognition and computer vision'' class.}

\cventry{April 2018 --- Sept. 2018}{Inria Willow}{Paris}{Research intern}{}{
  % \textit{Learning robotic skills by watching videos of humans.} Supervised by Josef Sivic and Ivan Laptev.
  % \begin{itemize}
    Trained robotics policies in simulation with reinforcement learning and time contrastive networks.
    %Visual rewards are extracted from real videos of human demonstrations using time contrastive networks.
  % \end{itemize}
}{}

%\cventry{Dec. 2016 --- July 2017}{MIP Robotics}{Paris}{Industrial partner}{}{
  %Project presented at the ``Agrégation''.
  % \begin{itemize}
  % Implementation of a motor control algorithm on a
% microcontroller for a collaborative robot.%
  % \end{itemize}
%}{}%


\cventry{May 2016 --- July 2016}{Prophesee}{Paris}{Research intern}{}{
  Implemented a super-resolution algorithm for an event-based camera.
  %\begin{itemize}
  %\item Added an IMU component in the FPGA, micro-controller and
    %libraries (Python/C++) of the existing event stream.
  %\item $2\times$ resolution increase was presented to clients and investors.
  %\end{itemize}
}

\cventry{2016}{École Normale Supérieure de Cachan}{Cachan}{President of the robotics club}{}{
%   % president in 2016 (15 members).
%   \begin{itemize}
%   \item Development of algorithms for mobile autonomous robots at the
%      french robotic competition.
% \begin{itemize}
%  \item Developed mobile autonomous robots and participated in the national robotic competition.
% \item Trained the 15 members and established a sponsorship with Faulhaber (2k€).
% \end{itemize}
%   % \item Responsible for the initiation and training of newcomers to robotics. 
%   % \item I established a  €2k sponsorship with Faulhaber.
%   \end{itemize}
}

% \cventry{2014 --- 2016}{School projects}{ENS Cachan}{}{}{
%   \begin{itemize}
%   \item \textit{Robot positioning system inspired by the VOR used in avionics (individual).} Simulation and implementation of an EKF, conception of a wireless communication system. I built a working prototype.
%   \item  \textit{Soft Robot (team of 6).} Pneumatic control of a soft structure.
%   \end{itemize}
% }

\vspace{-.3em}
\section{Education}


\cventry{2018 --- 2023}{Ph.D. in Computer Science}{École Normale Supérieure}{}{}{
% Computer Science.
  % Publications in computer vision and robotics: CVPR, ECCV, CoRL, RAL.
}

\cventry{2017 --- 2018}{M.Sc. in Applied mathematics}{École Normale Supérieure de Cachan}{}{}{
  ''MVA'': Mathematics, Computer Vision, Machine Learning. Obtained with highest honors.
  \newline
}

\cventry{2015 --- 2017}{M.Sc. in Electrical Engineering}{École Normale Supérieure de Cachan}{}{}{
  ''E3A'', ``FESUP'' programs. Obtained with highest honors. Ranked 1/70 (year 1), 1/13 (year 2).
}
% \cventry{2014 --- 2015}{B.Sc. Engineering}{École Normale Supérieure de Cachan}{}{}{
%   \textit{''SAPHIRE'' program.} Signal processing, mathematics, mechanical engineering, control, electronics.
% }

% \cventry{2012 --- 2014}{Preparatory class}{Lycée La Salle Passy Buzenval}{}{}{Program preparing for the PT national exam for entry to the French ``Grandes Écoles''.
% }

\section{Distinctions}
\cventry{2025}{Outstanding reviewer at CVPR 2025}{}{}{}{Award given to 711 out of 12593 reviewers.}
\cventry{2021}{Paper accepted as oral at CVPR 2021}{}{}{}{Top 4\% of 7500 submissions.}

\cventry{2020}{Winner of the BOP challenge, R6D Workshop at ECCV 2020}{}{}{}{
  CosyPose won 5 of the 6 awards, including best overall and best open-source method.
}
%\cventry{2018}{Granted a government-funded Ph.D. scholarship}{}{}{}{}
\cventry{2017}{Received at the Agrégation in Electrical Engineering}{}{}{}{Ranked 2/656.}
%\cventry{2014}{Received at École Normale Supérieure de Cachan}{}{}{}{
%  Granted with the status ``Élève Normalien'' and awarded a government-funded 4-years scholarship.

% \section{Skills}

% \cvitem{Languages}{
%   \begin{itemize}
%   \item French (native), English (professional, TOEIC).
%   \end{itemize}
% }

% \cvitem{Research}{
%     \begin{itemize}
%         \item Training state-of-the-art models using deep learning. 
%         \item Writing and reviewing papers.
%         \item Mentoring junior employees, interns, and PhD Students.
%         \item Engineering research infrastructures (training, evaluation, visualization) for solving novel practical problems through fast research iterations.
%     \end{itemize}
% }

% \cvitem{People}{
%   \begin{itemize}
%   \item Interviewing candidates.
%   \item Identifying technical problems in teams.
%   \end{itemize}
% }

% \cvitem{Computer}{
%   \begin{itemize}
%   \item Multi-node training (Linux, Slurm, Pytorch).
%   \item Python for scientific computing and deep learning: Pytorch, Numpy, Pandas, Dask, Ray.
%   \item Visualization: Matplotlib, Bokeh, Streamlit, LabelStudio, Datadog, Rerun.
%   \item Software development: Git, Github CI. 
%   \item Infrastructure: Docker, AWS EC2/Fargate.
%   \item Robotics software: ROS, Bullet, HPP, Pinocchio.
%   \item Synthetic data generation: Panda3D, Bullet, BlenderProc, Blender.
%   %\item Linux, GPU computing, distributed computing, networks, cluster usage (SLURM, NGC, SGE), cluster administration (SGE), git.
%   %\item Familiar with C, C++, MATLAB, VHDL and development on embedded systems (micro-controllers, FPGAs).
%   \item Figma, Adobe Premiere Pro.
%   \end{itemize}
% }

\section{Research}

\cvitem{Publications}{
Available on Google Scholar using this \underline{\color{blue}\href{https://scholar.google.com/citations?user=030pukwAAAAJ}{link}}.
}

\cvitem{Workshop Organization}{
  Co-organizer of the 7th and 8th International Workshop on Recovering 6D Object Pose at ECCV 2022 and ICCV 2023.
  %\item Co-organizer of the 8th International Workshop on Recovering 6D Object Pose, ICCV 2023.
}

\cvitem{Invited Talks}{
R6D Workshop (ICCV 2023, ECCV 2020), CIIRC CTU (2023, 2021), ENPC (2021), LAAS-CNRS~(2020).
}

\cvitem{Reviewer}{CVPR (2021, 2022, 2024, 2025), ICCV (2021, 2023), ECCV (2022), IROS (2020, 2021, 2022), ICRA (2022, 2022, 2023), TRO (2022, 2023), CDC (2021), RAL (2022), IJCV (2022), NeurIPS~(2020).}

% \cvitem{Reviewer}{
%   \begin{itemize}
%   \item CVPR 2021, 2022, 2024, 2025
%   \item ICLR 2023, 2024
%   \item ECCV/ICCV 2021, 2022, 2023
%   \item IROS 2020, 2021, 2022
%   \item ICRA 2020, 2022, 2023
%   \item TRO 2022, 2023
%   \item CDC 2021
%   \item RAL 2022
%   \item IJCV 2022
%   \item NeurIPS 2020
%   \end{itemize}
% }


% \cvitem{Presentations}{
%   \begin{itemize}
%   \item ICCV 2023, International workshop on recovering 6D object pose. \textit{Presentation of the winners of the BOP challenge on 6D pose estimation of unseen objects.}
%   \item Inria, April 2023. \textit{Pose estimation of rigid objects and robots.}
%   \item CIIRC CTU, May 2023. \textit{MegaPose: 6D Pose estimation of novel objects via Render \& Compare.}
%   % \item Inria Willow-Sierra-Prairie Seminar, October 2022, Saint-Raphael, France.\textit{Visual state estimation of scenes containing objects and robots.}
%   \item ENPC, April 2021. \textit{Pose estimation of rigid objects and articulated robots.}
%   % \item CIIRC CTU Intelligent Machine Perception (virtual), April 2021. \textit{Pose estimation of rigid objects and articulated robots.}
%   \item LAAS-CNRS, September 2020. \textit{Single-view and multi-view 6D object pose estimation and its applications in robotics.}
%   \item ECCV 2020, R6D Workshop. \textit{CosyPose, winning entry in the BOP challenge.}
%   % \item ENPC (virtual), April 2020. \textit{CosyPose: Consistent multi-view multi-object 6D pose estimation.}
%   \item RSS 2019, Workshop on Scalable Learning for Integrated Perception and Planning. \textit{Monte-Carlo Tree Search for Efficient Visually Guided Rearrangement Planning.}
%   % \item Willow --- ENPC --- Berkeley Workshop on Vision and Robotics, November 2019, Inria, Paris, France.\textit{Monte-Carlo Tree Search for Efficient Visually Guided Rearrangement Planning.}
%   % \item Willow --- Gepetto Workshop on Vision and Robotics, October 2019, LAAS-CNRS, Toulouse, France. \textit{Monte-Carlo Tree Search for Efficient Visually Guided Rearrangement Planning.}
%   % \item Inria Willow-Sierra Seminar, June 2019, Marseille, France. \textit{Reinforcement Learning: Applications and limitations.}
%   \end{itemize}
% }


%\cvitem{Mentorship}{Co-advising 2 PhD students (Georgy Ponimatkin and Martin Cifka) with Josef Sivic.}


% \section{Teaching}

%\nocite{*}
%\printbibliography[title={Publications}]


% \cvitem{Students}{
%   Co-advising 2 PhD students (Martin Cifka and Georgy Ponimatkin) with Josef Sivic
% }


% \section{Miscellaneous}
% \cvitem{Sports}{
%   \begin{itemize}
%   \item Regular practice of road cycling (FTP:\@ 4.1 W/kg in 2023).
%   %\item Occasional practice of sailing (windsurf, dinghy, catamaran).
%   \item Former high-level player in competitive online games (CS:GO, LoL, WoW).
%   \end{itemize}
% }

\end{document}